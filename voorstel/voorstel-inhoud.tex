%---------- Inleiding ---------------------------------------------------------

% TODO: Is dit voorstel gebaseerd op een paper van Research Methods die je
% vorig jaar hebt ingediend? Heb je daarbij eventueel samengewerkt met een
% andere student?
% Zo ja, haal dan de tekst hieronder uit commentaar en pas aan.

%\paragraph{Opmerking}

% Dit voorstel is gebaseerd op het onderzoeksvoorstel dat werd geschreven in het
% kader van het vak Research Methods dat ik (vorig/dit) academiejaar heb
% uitgewerkt (met medesturent VOORNAAM NAAM als mede-auteur).
% 

\section{Inleiding}%
\label{sec:inleiding}

De videogame-industrie is uitgegroeid tot een van de meest winstgevende sectoren binnen de entertainmentindustrie de afgelopen decennia. Het is zelfs zo gegroeid dat het de inkomsten overtreft van de film- en muziekindustrie. Deze groei gaat hand in hand met een toenemende complexiteit van games: niet-lineaire verhalen, multiplayerfunctionaliteiten en uitgebreide open werelden. Door deze complexiteit worden game-ontwikkelaars voortdurend op de proef gesteld op het gebied van kwaliteitsborging, waarbij de traditionele testmethoden meer onder druk komen te staan.



De doelgroep van dit onderzoek bestaat uit game-ontwikkelaars en quality assurance-teams, met focus op studios die met een beperkt budget en met veel tijdsdruk geconfronteerd worden. Zoals eerder vermeld is handmatig testen arbeidsintensief, vatbaar voor menselijke fouten en kostbaar. Het gevolg hiervan is dat het testproces een bottleneck vormt voor deze soort bedrijven. Daarentegen is automatisering moeilijk realiseerbaar, omdat games nu minder lineair en meer interactief zijn.



De probleemstelling die centraal staat in dit onderzoek is dat gametesten duur, tijdrovend en moeilijk volledig te automatiseren is door de complexiteit en niet-lineariteit van moderne games. Het is moeilijk voor een handmatige tester om alle verschillende scenario's en interacties te dekken. Dit zorgt ervoor dat testdekking inconsistent is en dat er een verhoogd risico is op bugs die pas na release ontdekt worden. Dit leidt toch tot mogelijke reputatieschade en ontevreden spelers. De centrale onderzoeksvraag luidt daarom: Hoe kan AI worden ingezet om speltesting efficiënter te maken zonder verlies aan testkwaliteit of bugdetectie?

%---------- Stand van zaken ---------------------------------------------------

\section{Literatuurstudie}%
\label{sec:literatuurstudie}

\subsection{Traditionele game testing}
Game testing verschilt fundamenteel van traditionele software testing door de niet-deterministische aard van gameplay en complexe interacties tussen systemen~\autocite{Politowski2021}. Handmatige testing blijft de dominante aanpak, hoewel dit gepaard gaat met hoge kosten en beperkte schaalbaarheid. Bestaande testing tools zijn vaak onvoldoende afgestemd op de specifieke behoeften van game development~\autocite{Politowski2022}. \textcite{Butt2023} tonen aan dat bepaalde categorieën bugs zoals collision detection en physics-gerelateerde problemen regelmatig gemist worden bij traditionele testmethoden. Geen universele oplossing bestaat; verschillende gamegenres vereisen specifieke testaanpakken~\autocite{Albaghajati2020}.

\subsection{Agent-based testing}
\textcite{Shirzadehhajimahmood2021} demonstreren dat agent-based testing veelbelovend is voor het systematisch verkennen van game states en het detecteren van edge cases. \textcite{Ariyurek2019} vergelijken synthetic agents (coverage-georiënteerd) met human-like agents (usability-gericht), wat wijst op het potentieel van een hybride aanpak. De IV4XR framework van \textcite{Prasetya2022} toonde aan dat AI-agents significant sneller kunnen testen dan menselijke testers, maar moeite hebben met het detecteren van subtiele gameplay mechanics en usability issues.

\subsection{Deep reinforcement learning}
Met deep reinforcement learning zijn agents in staat game environments zelfstandig te doorlopen zonder enige vorm van voorkennis. Vergelijkingen laten zien dat het Wuji-framework effectiever is in het opsporen van bugs dan traditionele scripted bots in online multiplayer games. In AAA-games stuiten implementaties echter op obstakels zoals uitgebreide trainingstijden en hoge eisen aan compute-resources. Reinforcement learning-agents hebben hun waarde bewezen voor zowel load testing als stress testing.

\subsection{Computer vision}
Computer vision methods bieden een alternatieve benadering door visuele game states te analyseren voor black-box testing~\autocite{Paduraru2021}. De keuze van AI-techniek hangt sterk af van gametype, ontwikkelingsstadium en beschikbare resources~\autocite{Zarembo2019}. Semi-automatische frameworks die menselijke expertise combineren met AI-capabilities worden beschouwd als essentieel~\autocite{Nantes2008}.

\subsection{Kennislacunes}
Bestaande literatuur biedt echter beperkte kwantitatieve vergelijkingen tussen AI-agents en menselijke testers in realistische industriële contexten. Dit onderzoek adresseert deze kennislacune door beide benaderingen systematisch te evalueren op testsnelheid, bugdetectie-effectiviteit en testdekking, met specifieke focus op praktische toepasbaarheid voor mid-size studios en indie-ontwikkelaars.

%---------- Methodologie ------------------------------------------------------
\section{Methodologie}%
\label{sec:methodologie}

Hier beschrijf je hoe je van plan bent het onderzoek te voeren. Welke onderzoekstechniek ga je toepassen om elk van je onderzoeksvragen te beantwoorden? Gebruik je hiervoor literatuurstudie, interviews met belanghebbenden (bv.~voor requirements-analyse), experimenten, simulaties, vergelijkende studie, risico-analyse, PoC, \ldots?

Valt je onderwerp onder één van de typische soorten bachelorproeven die besproken zijn in de lessen Research Methods (bv.\ vergelijkende studie of risico-analyse)? Zorg er dan ook voor dat we duidelijk de verschillende stappen terug vinden die we verwachten in dit soort onderzoek!

Vermijd onderzoekstechnieken die geen objectieve, meetbare resultaten kunnen opleveren. Enquêtes, bijvoorbeeld, zijn voor een bachelorproef informatica meestal \textbf{niet geschikt}. De antwoorden zijn eerder meningen dan feiten en in de praktijk blijkt het ook bijzonder moeilijk om voldoende respondenten te vinden. Studenten die een enquête willen voeren, hebben meestal ook geen goede definitie van de populatie, waardoor ook niet kan aangetoond worden dat eventuele resultaten representatief zijn.

Uit dit onderdeel moet duidelijk naar voor komen dat je bachelorproef ook technisch voldoen\-de diepgang zal bevatten. Het zou niet kloppen als een bachelorproef informatica ook door bv.\ een student marketing zou kunnen uitgevoerd worden.

Je beschrijft ook al welke tools (hardware, software, diensten, \ldots) je denkt hiervoor te gebruiken of te ontwikkelen.

Probeer ook een tijdschatting te maken. Hoe lang zal je met elke fase van je onderzoek bezig zijn en wat zijn de concrete \emph{deliverables} in elke fase?

%---------- Verwachte resultaten ----------------------------------------------
\section{Verwacht resultaat, conclusie}%
\label{sec:verwachte_resultaten}

Hier beschrijf je welke resultaten je verwacht. Als je metingen en simulaties uitvoert, kan je hier al mock-ups maken van de grafieken samen met de verwachte conclusies. Benoem zeker al je assen en de onderdelen van de grafiek die je gaat gebruiken. Dit zorgt ervoor dat je concreet weet welk soort data je moet verzamelen en hoe je die moet meten.

Wat heeft de doelgroep van je onderzoek aan het resultaat? Op welke manier zorgt jouw bachelorproef voor een meerwaarde?

Hier beschrijf je wat je verwacht uit je onderzoek, met de motivatie waarom. Het is \textbf{niet} erg indien uit je onderzoek andere resultaten en conclusies vloeien dan dat je hier beschrijft: het is dan juist interessant om te onderzoeken waarom jouw hypothesen niet overeenkomen met de resultaten.


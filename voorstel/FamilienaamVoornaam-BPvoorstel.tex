%==============================================================================
% Sjabloon onderzoeksvoorstel bachproef
%==============================================================================
% Gebaseerd op document class `hogent-article'
% zie <https://github.com/HoGentTIN/latex-hogent-article>

% Voor een voorstel in het Engels: voeg de documentclass-optie [english] toe.
% Let op: kan enkel na toestemming van de bachelorproefcoördinator!
\documentclass{hogent-article}

% Invoegen bibliografiebestand
\addbibresource{voorstel.bib}

% Informatie over de opleiding, het vak en soort opdracht
\studyprogramme{Professionele bachelor toegepaste informatica}
\course{Bachelorproef}
\assignmenttype{Onderzoeksvoorstel}
% Voor een voorstel in het Engels, haal de volgende 3 regels uit commentaar
% \studyprogramme{Bachelor of applied information technology}
% \course{Bachelor thesis}
% \assignmenttype{Research proposal}

\academicyear{2025-2026} % TODO: pas het academiejaar aan

% TODO: Werktitel
\title{Game testing met AI: Een onderzoek naar prestaties, schaalbaarheid en detectiekwaliteit}

% TODO: Studentnaam en emailadres invullen
\author{Nelis Dierckxsens}
\email{nelis.dierckxsens@student.hogent.be}

% TODO: Medestudent
% Gaat het om een bachelorproef in samenwerking met een student in een andere
% opleiding? Geef dan de naam en emailadres hier
% \author{Yasmine Alaoui (naam opleiding)}
% \email{yasmine.alaoui@student.hogent.be}

% TODO: Geef de co-promotor op
\supervisor[Co-promotor]{N/A (Mail, \href{mailto:n.a@mail.be}{n.a@mail.be})}

% Binnen welke specialisatierichting uit 3TI situeert dit onderzoek zich?
% Kies uit deze lijst:
%
% - Mobile \& Enterprise development
% - AI \& Data Engineering
% - Functional \& Business Analysis
% - System \& Network Administrator
% - Mainframe Expert
% - Als het onderzoek niet past binnen een van deze domeinen specifieer je deze
%   zelf
%
\specialisation{Mobile \& Enterprise development}
\keywords{Scheme, World Wide Web, $\lambda$-calculus}

\begin{document}

\begin{abstract}
Tegenwoordig worden games complexer en minder lineair. Daardoor wordt het testen van games, wat momenteel gebeurt met menselijke testers, zeer tijdrovend en duur is. Als we kijken naar het handmatig testen van games, zien we dat dat resulteert in ongelijke testdekking en een verhoging van de kans dat fouten niet ontdekt worden. Dit beïnvloedt de effectiviteit van het ontwikkelingsproces negatief. Dit onderzoek focust op op welke manier AI-agents gebruikt kunnen worden om gametesten effectiever te maken zonder de testdekking en het niet ontdekken van fouten negatief te beïnvloeden. Het resultaat is een vergelijkende analyse die aantoont of AI-bots sneller kunnen testen en beter fouten kunnen opsporen in vergelijking met menselijke testers. Daarnaast zal dit onderzoek de toepasbaarheid op verschillende gamegenres bekijken. De methode omvat een onderzoek naar beschikbare testtechnieken en AI-oplossingen 'autocite{Ariyurek2019, Prasetya2022}, gevolgd door een proof-of-concept waarbij een AI-testplatform wordt ontworpen en vergeleken met handmatige testresultaten, die zijn gebaseerd op elementen zoals snelheid, testdekking en het opsporen van fouten. Menselijke testers zijn effectiever in context doorgronden en edge-cases identificeren. AI-agents zijn daarentegen sneller en hebben een soortgelijke of zelfs betere bugdetectie. Deze resultaten geven duidelijke inzichten om testprocessen te verbeteren door gebruik te maken van zowel AI-agents, die dan routinetesten automatiseren, als menselijke testers, die zich dan focussen op ingewikkeldere scenario's. Door beiden toe te passen, resulteert dit in een kortere ontwikkelingstijd en een hogere productkwaliteit.\end{abstract}
\tableofcontents

% De hoofdtekst van het voorstel zit in een apart bestand, zodat het makkelijk
% kan opgenomen worden in de bijlagen van de bachelorproef zelf.
%---------- Inleiding ---------------------------------------------------------

% TODO: Is dit voorstel gebaseerd op een paper van Research Methods die je
% vorig jaar hebt ingediend? Heb je daarbij eventueel samengewerkt met een
% andere student?
% Zo ja, haal dan de tekst hieronder uit commentaar en pas aan.

%\paragraph{Opmerking}

% Dit voorstel is gebaseerd op het onderzoeksvoorstel dat werd geschreven in het
% kader van het vak Research Methods dat ik (vorig/dit) academiejaar heb
% uitgewerkt (met medesturent VOORNAAM NAAM als mede-auteur).
% 

\section{Inleiding}%
\label{sec:inleiding}

De videogame-industrie is uitgegroeid tot een van de meest winstgevende sectoren binnen de entertainmentindustrie de afgelopen decennia. Het is zelfs zo gegroeid dat het de inkomsten overtreft van de film- en muziekindustrie. Deze groei gaat hand in hand met een toenemende complexiteit van games: niet-lineaire verhalen, multiplayerfunctionaliteiten en uitgebreide open werelden. Door deze complexiteit worden game-ontwikkelaars voortdurend op de proef gesteld op het gebied van kwaliteitsborging, waarbij de traditionele testmethoden meer onder druk komen te staan.



De doelgroep van dit onderzoek bestaat uit game-ontwikkelaars en quality assurance-teams, met focus op studios die met een beperkt budget en met veel tijdsdruk geconfronteerd worden. Zoals eerder vermeld is handmatig testen arbeidsintensief, vatbaar voor menselijke fouten en kostbaar. Het gevolg hiervan is dat het testproces een bottleneck vormt voor deze soort bedrijven. Daarentegen is automatisering moeilijk realiseerbaar, omdat games nu minder lineair en meer interactief zijn.



De probleemstelling die centraal staat in dit onderzoek is dat gametesten duur, tijdrovend en moeilijk volledig te automatiseren is door de complexiteit en niet-lineariteit van moderne games. Het is moeilijk voor een handmatige tester om alle verschillende scenario's en interacties te dekken. Dit zorgt ervoor dat testdekking inconsistent is en dat er een verhoogd risico is op bugs die pas na release ontdekt worden. Dit leidt toch tot mogelijke reputatieschade en ontevreden spelers. De centrale onderzoeksvraag luidt daarom: Hoe kan AI worden ingezet om speltesting efficiënter te maken zonder verlies aan testkwaliteit of bugdetectie?

%---------- Stand van zaken ---------------------------------------------------

\section{Literatuurstudie}%
\label{sec:literatuurstudie}

Hier beschrijf je de \emph{state-of-the-art} rondom je gekozen onderzoeksdomein, d.w.z.\ een inleidende, doorlopende tekst over het onderzoeksdomein van je bachelorproef. Je steunt daarbij heel sterk op de professionele \emph{vakliteratuur}, en niet zozeer op populariserende teksten voor een breed publiek. Wat is de huidige stand van zaken in dit domein, en wat zijn nog eventuele open vragen (die misschien de aanleiding waren tot je onderzoeksvraag!)?

Je mag de titel van deze sectie ook aanpassen (literatuurstudie, stand van zaken, enz.). Zijn er al gelijkaardige onderzoeken gevoerd? Wat concluderen ze? Wat is het verschil met jouw onderzoek?

Verwijs bij elke introductie van een term of bewering over het domein naar de vakliteratuur, bijvoorbeeld~\autocite{Hykes2013}! Denk zeker goed na welke werken je refereert en waarom.

Draag zorg voor correcte literatuurverwijzingen! Een bronvermelding hoort thuis \emph{binnen} de zin waar je je op die bron baseert, dus niet er buiten! Maak meteen een verwijzing als je gebruik maakt van een bron. Doe dit dus \emph{niet} aan het einde van een lange paragraaf. Baseer nooit teveel aansluitende tekst op eenzelfde bron.

Als je informatie over bronnen verzamelt in JabRef, zorg er dan voor dat alle nodige info aanwezig is om de bron terug te vinden (zoals uitvoerig besproken in de lessen Research Methods).

% Voor literatuurverwijzingen zijn er twee belangrijke commando's:
% \autocite{KEY} => (Auteur, jaartal) Gebruik dit als de naam van de auteur
%   geen onderdeel is van de zin.
% \textcite{KEY} => Auteur (jaartal)  Gebruik dit als de auteursnaam wel een
%   functie heeft in de zin (bv. ``Uit onderzoek door Doll & Hill (1954) bleek
%   ...'')

Je mag deze sectie nog verder onderverdelen in subsecties als dit de structuur van de tekst kan verduidelijken.

%---------- Methodologie ------------------------------------------------------
\section{Methodologie}%
\label{sec:methodologie}

Hier beschrijf je hoe je van plan bent het onderzoek te voeren. Welke onderzoekstechniek ga je toepassen om elk van je onderzoeksvragen te beantwoorden? Gebruik je hiervoor literatuurstudie, interviews met belanghebbenden (bv.~voor requirements-analyse), experimenten, simulaties, vergelijkende studie, risico-analyse, PoC, \ldots?

Valt je onderwerp onder één van de typische soorten bachelorproeven die besproken zijn in de lessen Research Methods (bv.\ vergelijkende studie of risico-analyse)? Zorg er dan ook voor dat we duidelijk de verschillende stappen terug vinden die we verwachten in dit soort onderzoek!

Vermijd onderzoekstechnieken die geen objectieve, meetbare resultaten kunnen opleveren. Enquêtes, bijvoorbeeld, zijn voor een bachelorproef informatica meestal \textbf{niet geschikt}. De antwoorden zijn eerder meningen dan feiten en in de praktijk blijkt het ook bijzonder moeilijk om voldoende respondenten te vinden. Studenten die een enquête willen voeren, hebben meestal ook geen goede definitie van de populatie, waardoor ook niet kan aangetoond worden dat eventuele resultaten representatief zijn.

Uit dit onderdeel moet duidelijk naar voor komen dat je bachelorproef ook technisch voldoen\-de diepgang zal bevatten. Het zou niet kloppen als een bachelorproef informatica ook door bv.\ een student marketing zou kunnen uitgevoerd worden.

Je beschrijft ook al welke tools (hardware, software, diensten, \ldots) je denkt hiervoor te gebruiken of te ontwikkelen.

Probeer ook een tijdschatting te maken. Hoe lang zal je met elke fase van je onderzoek bezig zijn en wat zijn de concrete \emph{deliverables} in elke fase?

%---------- Verwachte resultaten ----------------------------------------------
\section{Verwacht resultaat, conclusie}%
\label{sec:verwachte_resultaten}

Hier beschrijf je welke resultaten je verwacht. Als je metingen en simulaties uitvoert, kan je hier al mock-ups maken van de grafieken samen met de verwachte conclusies. Benoem zeker al je assen en de onderdelen van de grafiek die je gaat gebruiken. Dit zorgt ervoor dat je concreet weet welk soort data je moet verzamelen en hoe je die moet meten.

Wat heeft de doelgroep van je onderzoek aan het resultaat? Op welke manier zorgt jouw bachelorproef voor een meerwaarde?

Hier beschrijf je wat je verwacht uit je onderzoek, met de motivatie waarom. Het is \textbf{niet} erg indien uit je onderzoek andere resultaten en conclusies vloeien dan dat je hier beschrijft: het is dan juist interessant om te onderzoeken waarom jouw hypothesen niet overeenkomen met de resultaten.



\printbibliography[heading=bibintoc]

\end{document}